\documentclass[12pt, a4paper]{article}
%字体大小为12点,纸张大小为A4。
\usepackage[utf8]{inputenc}
%在inputenc中使用UTF-8编码,以支持中文和其他Unicode字符
\usepackage{ctex}
%引入ctex宏包,提供对中文的支持
\usepackage{amsmath, amssymb, amsthm}

% 设置页面布局
\usepackage[left=2.5cm, right=2.5cm, top=3cm, bottom=3cm]{geometry}

% 定义数学定理环境 这些都是自定义环境
\newtheorem{theorem}{定理}[section]
\newtheorem{definition}{定义}[section]
\newtheorem{example}{例子}[section]
\newtheorem{corollary}{推论}[section]
\newtheorem{lemma}{引理}[section]
\newtheorem{proposition}{命题}[section]

\newtheorem{case}{情形}[section]
\newtheorem{tips}{tip}[section]
\newtheorem{question}{问题}[section]
\usepackage{enumitem}

% 设置全局段落缩进
\setlist[1]{left=1em}   % 一级段落缩进为1em
\setlist[2]{left=2em}   % 二级段落缩进为2em
\setlist[3]{left=3em}   % 三级段落缩进为3em

% 文档开始
\begin{document}

\title{常微分方程}
\author{Xinyi}
\date{\today}
\maketitle

\section{奇解与包络}
\begin{case}不存在奇解  \\
    $ f_y(x, y)$存在且有界,
    也就是解的存在唯一性定理在定义区间上全部满足
\end{case}

\begin{case} 包络线及奇解的求法\\
    是包络线一定是奇解\\
    \begin{case} 
        C-判别
        $$
        \left\{
            \begin{array}{l}
        \Phi(x, y, c)=0 \\
        \Phi c^{\prime}(x, y, c)=0
        \end{array}
        \right.
        $$\\    
    \end{case}
    \begin{case} 
        p-判别  
        $$
        \left\{
            \begin{array}{l}
        F(x, y, p)=0 \\
        Fp^{\prime}(x, y, p)=0
        \end{array}
        \right.
        $$\\
    \end{case}
\end{case}

\section{平面定性理论}
只学平面系统,易于对特征根情况分类
\section{课后题背诵}
\begin{example}
在条形区域 $a \leqslant x \leqslant b,|y|<+\infty$ 上, 
假设方程 $\frac{dy}{dx} = f(x, y)$的所有解都唯一, 对其中任意两个解 $y_1(x), y_2(x)$, 
如果有 $y_1\left(x_0\right)<y_2\left(x_0\right)$, 
则必有 $y_1(x)<y_2(x), x_0 \leqslant x \leqslant b$.
\end{example}
\begin{proof}
    反证法:假设存在$x{\prime}$,
    使得 \(y_1\left(x'\right) \geq y_2\left(x'\right)\)。\\
    若第一种情况 \(y_1\left(x'\right) = y_2\left(x'\right) = y'\),\\
    而根据解的存在唯一性,知过点 \((x', y')\) 的解应为唯一解,矛盾。\\
    若第二种情况 \(y_1\left(x'\right) > y_2\left(x'\right)\),\\
    而根据连续性,可知 \(y_1\left(x'\right) = y_2\left(x'\right)\) 矛盾。    
    故假设不成立,证毕。
\end{proof}
\begin{example}
    .3 1. 试证明: 对任意的 $x_0$ 及满足条件 $0<y_0<1$ 的 $y_0$, 
    方程 $\frac{\mathrm{d} y}{\mathrm{~d} x}=\frac{y(y-1)}{1+x^2+y^2}$ 的满足条件 $y\left(x_0\right)=y_0$ 的解
     $y=y(x)$ 在 $(-\infty,+\infty)$ 上存在.
\end{example}   
\begin{proof}
   首先找特解:y=0,y=1是方程在$(-\infty,+\infty)$上的特解。\\
   \begin{definition}特解
    不依赖于任何常数
    \end{definition}
    易知方程的右端函数满足解的延展定理和存在唯一性定理条件\\
    \begin{question}
        延展定理和存在唯一性定理的条件究竟一样不一样
    \end{question}
    现在考虑过初值\(x_0, y_0\)的解,根据唯一性,该解不能穿过直线y=0和y=1。\\
    因此只能向左右两侧延展,从而初值解在\(-\infty,+\infty\)上存在。
\end{proof}
\begin{example}
    设
\end{example}
\begin{definition}
    这是一个定义的例子。
\end{definition}

\begin{theorem}
    这是一个定理的例子。
\end{theorem}

\section{证明}
提供一些证明的例子。

\begin{proof}
    这是一个证明的例子。
\end{proof}

\section{例子}
展示一些数学例子。

\begin{example}
    这是一个例子。
\end{example}

\section{一边学latex一边学常微分}

\begin{tips}
自定义环境  
\\newenvironment{定义name}{显示name}[num]{before}{after}  
\end{tips}
\begin{tips}
如果你想在文本中显示一个反斜杠,可以使用两个反斜杠 \\ \\
同时,两个反斜杠也有换行作用
\end{tips}
\begin{tips}
    一些待办,缩进设置
\end{tips}
% 文档结束
\end{document}
