\documentclass[12pt, a4paper]{article}
%字体大小为12点,纸张大小为A4。
\usepackage[utf8]{inputenc}
%在inputenc中使用UTF-8编码,以支持中文和其他Unicode字符
\usepackage{ctex}
%引入ctex宏包,提供对中文的支持
\usepackage{amsmath, amssymb, amsthm}

% 设置页面布局
\usepackage[left=2.5cm, right=2.5cm, top=3cm, bottom=3cm]{geometry}

% 定义数学定理环境 这些都是自定义环境
\newtheorem{theorem}{定理}[section]
\newtheorem{definition}{定义}[section]
\newtheorem{example}{例子}[section]
\newtheorem{corollary}{推论}[section]
\newtheorem{lemma}{引理}[section]
\newtheorem{proposition}{命题}[section]

\newtheorem{case}{情形}[section]
\newtheorem{tips}{tip}[section]

\usepackage{enumitem}

% 设置全局段落缩进
\setlist[1]{left=1em}   % 一级段落缩进为1em
\setlist[2]{left=2em}   % 二级段落缩进为2em
\setlist[3]{left=3em}   % 三级段落缩进为3em

% 文档开始
\begin{document}

\title{常微分方程}
\author{Xinyi}
\date{\today}
\maketitle

\section{奇解与包络}
\begin{case}不存在奇解  \\
    $ f_y(x, y)$存在且有界,
    也就是解的存在唯一性定理在定义区间上全部满足
\end{case}

\begin{case} 包络线及奇解的求法\\
    是包络线一定是奇解\\
    \begin{case} 
        C-判别
        $$
        \left\{
            \begin{array}{l}
        \Phi(x, y, c)=0 \\
        \Phi c^{\prime}(x, y, c)=0
        \end{array}
        \right.
        $$\\    
    \end{case}
    \begin{case} 
        p-判别  
        $$
        \left\{
            \begin{array}{l}
        F(x, y, p)=0 \\
        F c^{\prime}(x, y, p)=0
        \end{array}
        \right.
        $$\\
    \end{case}
\end{case}

\section{基本概念}
介绍一些基本的数学概念和定义。

\begin{definition}
    这是一个定义的例子。
\end{definition}

\begin{theorem}
    这是一个定理的例子。
\end{theorem}

\section{证明}
提供一些证明的例子。

\begin{proof}
    这是一个证明的例子。
\end{proof}

\section{例子}
展示一些数学例子。

\begin{example}
    这是一个例子。
\end{example}

\section{一边学latex一边学常微分}

\begin{tips}
自定义环境  
\\newenvironment{定义name}{显示name}[num]{before}{after}  
\end{tips}
\begin{tips}
如果你想在文本中显示一个反斜杠,可以使用两个反斜杠 \\ \\
同时,两个反斜杠也有换行作用
\end{tips}
% 文档结束
\end{document}
